\documentclass[../main.tex]{subfiles}

\begin{document}

\begin{itemize}
    \item[s.~1] Affine schemes.
    \item[s.~2] Preschemes and morphisms of preschemes.
    \item[s.~3] Products of preschemes.
    \item[s.~4] Sub-preschemes and immersion maps.
    \item[s.~5] Reduced preschemes; separation condition.
    \item[s.~6] Finiteness conditions.
    \item[s.~7] Rational maps.
    \item[s.~8] Chevalley schemes.
    \item[s.~9] Details on quasi-coherent sheaves.
    \item[s.~10] Formal schemes.
\end{itemize}

\oldpage{79}Sections 1 to 8 intend only to develop a language, which will be used in all that follows.
We note, however, that following the general spirit of this Treatise, sections 7 and 8 will be less used than the others, and in a less essential manner; we speak of Chevalley schemes only in order to be able to link to the language of Chevalley [1] and Nagata [9].
Section 9 gives some definitions and results about quasi-coherent sheaves

TODO

\end{document}
