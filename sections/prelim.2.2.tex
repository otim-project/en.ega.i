\documentclass[../main.tex]{subfiles}

\begin{document}

\begin{cx}{2.2.1}
\oldpage{23}We say that a topological space $X$ is \emph{Noetherian} if the set of open
subsets of $X$ satisfies the \emph{maximal} condition, or, equivalently, if the set of closed
subsets of $X$ satisfies the \emph{minimal} condition. We say that $X$ is
\emph{locally Noetherian} if all $x\in X$ admit a neighborhood which is a Noetherian subspace.
\end{cx}

\begin{cx}{2.2.2}
Let $E$ be an ordered set satisfying the \emph{minimal} condition, and let $\mathbf{P}$ be a
property of the elements of $E$ subject to the following condition: if $a\in E$ is such that
for any $x<a$, $\mathbf{P}(x)$ is true, then $\mathbf{P}(a)$ is true. Under these conditions,
$\mathbf{P}(x)$ \emph{is true for all} $x\in E$
(``principle of Noetherian recurrence''). Indeed, let $F$ be the set of $x\in E$ for
which $\mathbf{P}(x)$ is false; if $F$ were not empty, it would have a minimal element $a$,
and as then $\mathbf{P}(x)$ is true for all $x<a$, $\mathbf{P}(a)$ would be true, which is
a contradiction.

We will apply this principle in particular when $E$ is a
\emph{set of closed subsets of a Noetherian space}.
\end{cx}

\begin{cx}{2.2.3}
Any subspace of a Noetherian space is Noetherian. Conversely,
any topological space that is a finite union of Noetherian subspaces is Noetherian.
\end{cx}

\begin{cx}{2.2.4}
Any Noetherian space is quasi-compact; conversely, any  topological space in which all
open sets are quasi-compact is Noetherian.
\end{cx}

\begin{cx}{2.2.5}
A Noetherian space has only a \emph{finite} number of irreducible components,
as we see by Noetherian recurrence.
\end{cx}

\end{document}

