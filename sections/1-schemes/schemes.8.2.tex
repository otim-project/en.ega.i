\documentclass[../main.tex]{subfiles}

\begin{document}

\begin{cx}{8.2.1}
    Let $X$ be an \emph{integral} prescheme, and $R$ its field of rational functions, identical to the local ring of the generic point $a$ of $X$; for all $x\in X$, we know that $\O_x$ can be canonically identified with a subring of $R$ (7.1.5), and for every rational function $f\in R$, the domain of definition $\delta(f)$ of $f$ is the open set of $x\in X$ such that $f\in\O_x$.
    It thus follows (7.2.6) that, for every open $U\subset X$, we have
    \begin{equation*}
        \Gamma(U,\O_X) = \bigcap_{x\in U}\O_x.\tag{8.2.1.1}
    \end{equation*}
\end{cx}

\begin{cx}[Proposition]{8.2.2}
    \oldpage{166}Let $X$ be an integral prescheme, and $R$ its field of rational fractions.
    For $X$ to be a scheme, it is necessary and sufficient that the relation "$\O_x$ and $\O_y$ are allied" (8.1.4), for points $x,y$ of $X$, implies that $x=y$.
\end{cx}

\begin{cx}[Corollary]{8.2.3}
    Let $X$ be an integral scheme, and $x,y$ two points of $X$.
    In order that $x\in\overline{\{y\}}$, it is necessary and sufficient that $\O_x\subset\O_y$, or, equivalently, that every rational function defined at $x$ is also defined at $y$.
\end{cx}

\begin{cx}[Corollary]{8.2.4}
    If $X$ is an integral scheme then the map $x\to\O_x$ is injective; equivalently, if $x$ and $y$ are two distinct points of $X$, then there exists a rational function defined at one of these points but not the other.
\end{cx}

\oldpage{167}This follows from (8.2.3) and the axiom ($\mathrm{T}_0$) (2.1.4).

\begin{cx}[Corollary]{8.2.5}
    Let $X$ be an integral scheme whose underlying space is Noetherian; letting $f$ run over the field $R$ of rational functions on $X$, the sets $\delta(f)$ generate the topology of $X$.
\end{cx}

\begin{cx}{8.2.6}
\end{cx}

\begin{cx}[Proposition]{8.2.7}
    Let $X,Y$ be two integral schemes, $f\colon X\to Y$ a dominant morphism (2.2.6), and $K$ (resp. $L$) the field of rational functions on $X$ (resp. $Y$).
    Then $L$ can be identified with a sub-field of $K$, and for all $x\in X$, $\O_{f(x)}$ is the unique local ring of $Y$ dominated by $\O_x$.
\end{cx}

\begin{cx}[Proposition]{8.2.8}
    Let $X$ be an \emph{irreducible} prescheme; and $f\colon X\to Y$ a local immersion (\emph{resp.} a local isomorphism); and suppose further that $f$ is separated.
    Then $f$ is an immersion (\emph{resp.} an open immersion).
\end{cx}

\oldpage{168}a local immersion (resp. a local isomorphism)

\unsure{TODO}

\end{document}
