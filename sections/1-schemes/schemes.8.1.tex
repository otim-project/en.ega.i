\documentclass[../main.tex]{subfiles}

\begin{document}

For every local ring $A$, we denote by $\mathfrak{m}(A)$ the maximal ideal of $A$.

\begin{cx}[Lemma]{8.1.1}
    Let $A$ and $B$ be two local rings such that $A\subset B$; then the following conditions are equivalent: \emph{(i)} $\mathfrak{m}(B)\cap A=\mathfrak{m}(A)$; \emph{(ii)} $\mathfrak{m}(A)\subset\mathfrak{m}(B)$; \emph{(iii)} $1$ is not an element of the ideal of $B$ generated by $\mathfrak{m}(A)$.
\end{cx}

It's evident that (i) implies (ii), and (ii) implies (iii); lastly, if (iii) is true, then $\mathfrak{m}(B)\cap A$ contains $\mathfrak{m}(A)$ and doesn't contain $1$, and is thus equal to $\mathfrak{m}(A)$.

When the equivalent conditions of (8.1.1) are satisfied, we say that $B$ \emph{dominates} $A$; this is equivalent to saying that the injection $A\to B$ is a \emph{local} homomorphism.
It is clear that, in the set of local subrings of a ring $R$, the relation given by domination is an \unsure{order}.

\begin{cx}{8.1.2}
    Now consider a \emph{field} $R$.
    For all subrings $A$ of $R$, we denote by $L(A)$ the set of local rings $A_\mathfrak{p}$, where $\mathfrak{p}$ runs over the prime spectrum of $A$; they are identified with the subrings of $R$ containing $A$.
    Since $\mathfrak{p}=(\mathfrak{p}A_\mathfrak{p})\cap A$, the map $\mathfrak{p}\to A_\mathfrak{p}$ from $\Spec(A)$ into $L(A)$ is bijective.
\end{cx}

\begin{cx}[Lemma]{8.1.3}
    Let $R$ be a field, and $A$ a subring of $R$.
    For a local subring $M$ of $R$ to dominate a ring $A_\mathfrak{p}\in L(A)$ it is necessary and sufficient that $A\subset M$; the local ring $A_\mathfrak{p}$ dominated by $M$ is then unique, and corresponds to $\mathfrak{p}=\mathfrak{m}(M)\cap A$.
\end{cx}

Indeed, if $M$ dominates $A_\mathfrak{p}$, then $\mathfrak{m}(M)\cap A_\mathfrak{p}=\mathfrak{p}A_\mathfrak{p}$, by (8.1.1), whence the uniqueness of $\mathfrak{p}$; on the other hand, if $A\subset M$, then $\mathfrak{m}M\cap A=\mathfrak{p}$ is prime in $A$, and since $A\setminus\mathfrak{p}\subset M$, we have that $A_\mathfrak{p}\subset M$ and $\mathfrak{p}A_\mathfrak{p}\subset\mathfrak{m}(M)$, so $M$ dominates $A_\mathfrak{p}$

\begin{cx}[Lemma]{8.1.4}
    \oldpage{165}Let $R$ be a field, $M$ and $N$ two local subrings of $R$, and $P$ the subring of $R$ generated by $M\cup N$.
    Then the following conditions are equivalent:
    \begin{enumerate}[label=(\roman*)]
        \item There exists a prime ideal $\mathfrak{p}$ of $P$ such that $\mathfrak{m}(M)=\mathfrak{p}\cap M$ and $\mathfrak{m}(N)=\mathfrak{p}\cap N$.
        \item The ideal $\mathfrak{a}$ generated in $P$ by $\mathfrak{m}(M)\cup\mathfrak{m}(N)$ is distinct from $P$.
        \item There exists a local subring $Q$ of $R$ simultaneously dominating both $M$ and $N$.
    \end{enumerate}
\end{cx}

It is clear that (i) implies (ii); conversely, if $\mathfrak{a}\neq P$, then $\mathfrak{a}$ is contained in a maximal ideal $\mathfrak{n}$ of $P$, and since $1\not\in\mathfrak{n}$, $\mathfrak{n}\cap M$ contains $\mathfrak{m}(M)$ and is distinct from $M$, so $\mathfrak{n}\cap M=\mathfrak{m}(M)$, and similarly $\mathfrak{n}\cap N=\mathfrak{m}(N)$.
It is clear that, if $Q$ dominates both $M$ and $N$, then $P\subset Q$ and $\mathfrak{m}(M)=\mathfrak{m}(Q)\cap M=(\mathfrak{m}(Q)\cap P)\cap M$, and $\mathfrak{m}(N)=(\mathfrak{m}(Q)\cap P)\cap N$, so (iii) implies (i); the reciprocal is evident when we take $Q=P_\mathfrak{p}$.

When the conditions of (8.1.4) are satisfied, we say, with C.~Chevalley, that the local rings $M$ and $N$ are \emph{allied}.

\begin{cx}[Proposition]{8.1.5}
    Let $A$ and $B$ be two subrings of a field $R$, and $C$ the subring of $R$ generated by $A\cup B$.
    Then the following conditions are equivalent:
    \begin{enumerate}[label=(\roman*)]
        \item For every local ring $Q$ containing $A$ and $B$, we have that $A_\mathfrak{p}=B_\mathfrak{q}$, where $\mathfrak{p}=\mathfrak{m}(Q)\cap A$ and $\mathfrak{q}=\mathfrak{m}(Q)\cap B$.
        \item For all prime ideals $\mathfrak{r}$ of $C$, we have that $A_\mathfrak{p}=B_\mathfrak{q}$, where $\mathfrak{p}=\mathfrak{r}\cap A$ and $\mathfrak{q}=\mathfrak{r}\cap B$.
        \item If $M\in L(A)$ and $N\in L(B)$ are allied, then they are identical.
        \item $L(A)\cap L(B)=L(C)$.
    \end{enumerate}
\end{cx}

Lemmas (8.1.3) and (8.1.4) prove that (i) and (iii) are equivalent; it is clear that (i) implies (ii) by taking $Q=C_\mathfrak{r}$; conversely, (ii) implies (i), because if $Q$ contains $A\cup B$ then it contains $C$, and if $\mathfrak{r}=\mathfrak{m}(Q)\cap C$ then $\mathfrak{p}=\mathfrak{r}\cap A$ and $\mathfrak{q}=\mathfrak{r}\cap B$, from (8.1.3).
It is immediate that (iv) implies (i), because if $Q$ contains $A\cup B$ then it dominates a local ring $C_\mathfrak{r}\in L(C)$ by (8.1.3); by hypothesis we have that $C_\mathfrak{r}\in L(A)\cap L(B)$, and (8.1.1) and (8.1.3) prove that $C_\mathfrak{r}=A_\mathfrak{p}=B_\mathfrak{q}$.
We prove finally that (iii) implies (iv).
Let $Q\in L(C)$; $Q$ dominates some $M\in L(A)$ and some $N\in L(B)$ (8.1.3), so $M$ and $N$, being allied, are identical by hypothesis.
As we then have that $C\subset M$, we know that $M$ dominates some $Q'\in L(C)$ (8.1.3), so $Q$ dominates $Q'$, whence necessarily (8.1.3) $Q=Q'=M$, so $Q\in L(A)\cap L(B)$.
Conversely, if $Q\in L(A)\cap L(B)$, then $C\subset Q$, so (8.1.3) $Q$ dominates some $Q''\in L(C)\subset L(A)\cap L(B)$; $Q$ and $Q''$, being allied, are identical, so $Q''=Q\in L(C)$, which completes the proof.

\end{document}
