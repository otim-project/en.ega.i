\documentclass[../main.tex]{subfiles}

\begin{document}

\begin{cx}{2.1.1}
    Given a ringed space $(X,\O_X)$, we say that an open subset $V$ of $X$ is an \emph{affine open} if the ringed space $(V,\O_X|V)$ is an affine scheme (1.7.1).
\end{cx}

\begin{cx}[Definition]{2.1.2}
    We define a prescheme to be a ringed space $(X,\O_X)$ such that every point of $X$ admits an affine open neighbourhood.
\end{cx}

\begin{cx}[Proposition]{2.1.3}
    \oldpage{98}If $(X,\O_X)$ is a prescheme then the affine opens give a base for the topology of $X$.
\end{cx}

In effect, if $V$ is an arbitrary open neighbourhood of $x\in X$, then there exists by hypothesis an open neighbourhood $W$ of $x$ such that $(W,\O_X|W)$ is an affine scheme; we write $A$ to mean its ring.
In the space $W$, $V\cap W$ is an open neighbourhood of $x$; thus there exists $f\in A$ such that $D(f)$ is an open neighbourhood of $x$ contained inside $V\cap W$ (1.1.10 (i)).
The ringed space $(D(f),\O_X|D(f))$ is thus an affine scheme, isomorphic to $A_f$ (1.3.6), whence the proposition.

\begin{cx}[Proposition]{2.1.4}
    The underlying space of a prescheme is a Kolmogoroff space.
\end{cx}

In effect, if $x,y$ are two distinct points of a prescheme $X$ then it is clear that there exists an open neighbourhood of one of these points that does not contain the other if $x$ and $y$ are not in the same affine open; and if they are in the same affine open, this is a result of (1.1.8).

\begin{cx}[Proposition]{2.1.5}
    If $(X,\O_X)$ is a prescheme then every closed irreducible subset of $X$ admits exactly one generic point, and the map $x\mapsto\overline{\{x\}}$ is thus a bijection of $X$ onto its set of closed irreducible subsets.
\end{cx}

In effect, if $Y$ is a closed irreducible subset of $X$ and $y\in Y$, and if $U$ is an open affine neighbourhood of $y$ in $X$, then $U\cap Y$ is everywhere dense in $Y$, as well as irreducible (\textbf{0},~2.1.1 and 2.1.4); thus by (1.1.14), $U\cap Y$ is the closure in $U$ of a point $x$, and then $Y\subset\overline{U}$ is the closure of $x$ in $X$.
The uniqueness of the generic point of $X$ is a result of (2.1.4) and (\textbf{0},~2.1.3).

\begin{cx}{2.1.6}
    If $Y$ is a closed irreducible subset of $X$ and $y$ its generic point then the local ring $\O_y$, also written $\O_{X/Y}$, is called the \emph{local ring of $X$ along $Y$}, or the \emph{local ring of $Y$ in $X$}.

    If $X$ itself is irreducible and $x$ its generic point then we say that $\O_x$ is the \emph{ring of rational functions on $X$} (cf.~s.~7).
\end{cx}

\begin{cx}[Proposition]{2.1.7}
    If $(X,\O_X)$ is a prescheme then the ringed space $(U,\O_X|U)$ is a prescheme for every open subset $U$.
\end{cx}

    This follows directly from definition~(2.1.2) and proposition~(2.1.3).

    We say that $(U,\O_X|U)$ is the prescheme \emph{induced} on $U$ by $(X,\O_X)$, or the \emph{restriction} of $(X,\O_X)$ to~$U$.

\begin{cx}{2.1.8}
    We say that a prescheme $(X,\O_X)$ is \emph{irreducible} (resp. \emph{connected}) if the underlying space $X$ is irreducible (resp. connected).
    We say that a prescheme is \emph{integral} if it is \emph{irreducible and reduced} (cf.~(5.1.4)).
    We say that a prescheme $(X,\O_X)$ is \emph{locally integral} if every $x\in X$ admits an open neighbourhood $U$ such that the prescheme induced on $U$ by $(X,\O_X)$ is integral.
\end{cx}

\end{document}

